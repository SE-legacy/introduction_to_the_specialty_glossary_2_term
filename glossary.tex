\documentclass{article}
\usepackage{graphicx} % Required for inserting images
\usepackage[english, russian]{babel} 
\title{Глосарий}
\author{Егор Смирнов}
\date{May 2024}

\begin{document}


\section{Лекции №1, 2. Александр Кузнецов.}
\begin{enumerate}
    \item outsource "--- условный найм людей из других регионов / стран для экономии над з.п.
    \item Lead "--- ведущий разработчик программного обеспечения
    \item Cheif "---
    \item Senior "---
    \item Junior "---
    \item Middle "---
    \item Интерн "---
    \item ИИ "--- Искуственный Интелект
    \item Тестировщик "---
    \item Back"=end "---
    \item Front"=End "---
    \item Архитектор "---
    \item Кеш "---
    \item База Данных "---
    \item .NET framework
    \item API "--- программный интерфейс для предложения
\end{enumerate}

\section{Лекция №3. Игорь Юрьевич.}
\begin{enumerate}
    \item Компьютерная Безопасность "---
    \item Вайпер "--- программа для очищения данных
    \item Wi"=Fi (Вай-фай) "---
    \item Троян "---
    \item Хакер "---
    \item wipe "--- средство невосстановимого удаленя информации на носителе
    \item proksi сервер "---
\end{enumerate}

\section{Лекция №4. Шлюпкин Павел}
\begin{enumerate}
    \item биг тех "--- 
    \item таски "--- 
    \item project manager "---
    \item product manager "--- 
    \item scrum master "---
    \item agile coach "---
    \item AI"=тренер "---
    \item xr "---
    \item vr "---
    \item DevRel "---
    \item дата аналитик "---
    \item бизнес и системный аналитик "---
    \item продуктовый аналитик "---
    \item маркетенговый аналитик "---
    \item ux/ui "---
\end{enumerate}

\section{Лекция №5. БараБанов Никита}
\begin{enumerate}
    \item soa "--- метод разработки ........
    \item оффер "--- 
    \item msa "--- 
    \item framework "--- 
    \item логирование "---
    \item микросервис "---
    \item микросервисная архитектура "---
    \item мнолитная архитектура "---
    \item ui "--- user interface "---
    \item business logic "---
    \item data interface "---
    \item database "---
    \item RESTful API (delete post put get) "---
    \item json "--- js option natation "---
    \item devops "---
    \item pipeline "---
    \item unit test "---
    \item auto test "---
    \item deploy "--- запуск проекта на платформу (на hosting, beta test)
    \item production (prod) "---
    \item predproduction (predprod) "---
    \item push "--- отправить
    \item eml(uml???????) "---
    \item мониториг "---
    \item тз (техническое задание) "--- 
    \item waterfall model "---
    \item agile model "---
    \item scrum "---
    \item demo "--- предпросмотр решения
    \item экстремальное программирование "--- каждый аспект возводится в  абсолют
    \item kaizen "--- регулярное совершенствование 
\end{enumerate}

\section{Лекция 6. Михаил Ильичев}
\begin{enumerate}
    \item среды разработки "--- 
    \item git "---  система контроля версий
    \item unit"=test "---  
    \item SRE "--- 
\end{enumerate}

\section{Лекция 7. Иван Жадаев}
\begin{enumerate}
    \item sql запросы
\end{enumerate}


\section{Лекция 8. Вера Накончная}
\begin{enumerate}
    \item диограмма состояния "---
    \item диаграмма .... "--- 
    \item алгоритмическое мышление "---
    \item брокер сообщений "---
    \item swagger "--- один из часто используемых мини-справочников методов(спецификации)
    \item спецификации "--- справочник методов
    \item синхроная .... "---
    \item ассинхронная .. "---
    \item реалиционные бд "---
    \item nosql бд "---
    \item веб-API "---
\end{enumerate}

\section{Лекция 9. Александра Крючкова/Константин Ланских/Антона Пудикова}
\begin{enumerate}
    \item тестирование "---
    \item тестировщики "---
    \item sft инженеры (те кто тестят ручками) "---
\end{enumerate}
\end{document}
