\documentclass{article}
\usepackage{graphicx} % Required for inserting images

\usepackage{fontspec}
\usepackage[T2A]{fontenc}
\usepackage[utf8x]{inputenc}
\usepackage[russian]{babel} 

\setmainfont[Ligatures=TeX]{Times New Roman} %Шрифт
\setmonofont{Consolas}



\title{Глосарий}
\author{Егор Смирнов}
\date{May 2024}

\begin{document}
\begin{center}
    \textbf{Глоссарий}
\end{center}
\hspace{0.065\textwidth}
\textbf{A} 
\begin{enumerate}
    %textbf{A}
    \item {Architect(архитектор) (Александр Кузнецов) "---}
    \item {API (Александр Кузнецов) "--- программный интерфейс для приложения}
    \item {AI (Александр Кузнецов) "--- искуссственный интелект}
    \item {AI"=coach(AI"=тренер) (Шлюпкин Павел)"---}
    \item {Agile coach (Шлюпкин Павел) "---}
    \item {Auto"=test (БараБанов Никита) "---}
    
    \textbf{B}
    \item {Back"=end (Александр Кузнецов) "---} 
    \item {Big Data (Александр Кузнецов) "---}
    \item {Big Tex (Шлюпкин Павел) "---}
    \item {Business logic (БараБанов Никита) "---}
    %beta test определние вроде не говорили, но оно упоминалось
    \item {Beta test (БараБанов Никита) "---}
    
    
    \textbf{C}
    \item {Cheif (Александр Кузнецов) "---}
    \item {Cache (Александр Кузнецов) "---}
    
    \textbf{D}
    \item {Database (БараБанов Никита) "---}
    \item {Data аналитик (Шлюпкин Павел) "--- }
    \item {Data interface (БараБанов Никита) "---}
    \item {Demo (БараБанов Никита) "--- предпросмотр решения}
    \item {DevRel (Шлюпкин Павел) "---}
    \item {Devops (БараБанов Никита) "---}
    \item {Deploy (БараБанов Никита) "--- запуск проекта на платформу (на hosting, beta test)}
    \item {Docker (БараБанов Никита) "---}
    \item{.NET framework (dot NET framework) (Александр Кузнецов) "---}
    

    \textbf{F}
    \item {Flash (Александр Кузнецов) "---}
    \item {Front"=end (Александр Кузнецов) "---}
    \item {Framewoek (БараБанов Никита) "---}

    \textbf{G}
    \item {git (Михаил Ильичев) "--- система контроя версий}

    \textbf{H}
    \item {Hard-skills (Шлюпкин Павел)}
    \item {Hosting (БараБанов Никита) "---}
    \item {HTML (БараБанов Никита) "--- язык гиппертекстовой разметки}

    \textbf{J}
    \item {Junior (Александр Кузнецов) "---}
    \item {json (БараБанов Никита) "--- js option "---}

    \textbf{K}
    \item {Kaizen (БараБанов Никита) "---}

    \textbf{L}
    \item {Lead (Александр Кузнецов)"---}

    \textbf{M}
    \item {Middle (Александр Кузнецов) "---}
    \item {MSA (Барабанов Никита) "---}

    \textbf{N}
    \item {noSQL database (Вера Наконечная) "---}

    \textbf{O}
    \item {Openshift (Барабанов Никита) "---}
    \item {Outsource (Александр Кузнецов) "--- условный найм людей из других регионов / стран для экономии над заработной платы}

    \textbf{P}
    
    \item {Proxy server (Игорь Юрин) "---}
    \item {Project manager (Шлюпкин Павел) "---}
    \item {Product manager (Шлюпкин Павел) "---}
    \item {Productoin (Барабанов Никита) "---}
    \item {Predproducton (predprod) (Барабанов Никита) "---}
    \item {Pipeline (Барабанов Никита) "--- }
    \item {Push (Барабанов Никита) "---}
    
    
    \textbf{R}
    \item {RESETful API (Барабанов Никита) "--- }

    \textbf{S}
    \item {Senior (Александр Кузнецов) "---}
    \item {Scrum master (Шлюпкин Павел) "---}
    
    \item {Scrum (Барабанов Никита) "--- }
    \item {SOA (Барабанов Никита) "--- метод разработки ..........}
    \item {Soft-skills (Шлюпкин Павел) "---}
    \item {SRE (Михаил Ильичев) "--- }
    \item {SQL запросы (Иван Жадаев) "--- }
    
    \item {Swagger (Вера Наконечная) "--- один из часто используемых мини"=справочников методов(спецификаций)}
    \item {SFT инженеры (Константин Ланских) "--- }

    \textbf{T}
    \item {Task (говорил именно слово таски, поэтому хз мб лучше в Русский алфавит перенести) (Шлюпкин Павел) "--- }

    \textbf{U}
    \item {UX (Шлюпкин Павел) "--- user experience "---}
    \item {UI (Барабанов Никита) "--- user interface "---}
    \item {Unit test (Барабанов Никита) "---}
    \item {UML (БараБанов Никита) "--- }

    \textbf{V}
    \item {VR (Шлюпкин Павел) "--- }

    \textbf{W}
    \item {Waterfall model (Барабанов Никита) "--- }
    \item {Web"=API (Вера Наконечная) "---}
    \item {Wi"=Fi (Игорь Юрин) "---}
    \item {Wipe (Игорь Юрин) "--- средство невосстановимого удаления информации на носителе}
    \item {Wiper (Игорь Юрин) "--- программа для очищения данных }
    
    
    \textbf{X}
    \item {XML (Александра Крючкова)}
    \item {(XR (Шлюпкин Павел) "---}
    
    %Русский
    \textbf{А}
    \item {Алгоритмическое мышление (Вера Наконечная) "---}
    \item {Ассинхронная ..... (Вера Наконечная) "---}

    
    \textbf{Б}
    \item {Бизнес аналитик (Шлюпкин Павел) "---}
    \item {Брокер сообщений (Вера Наконечная) "---}
    

    \textbf{Д}
    \item {Диаграмма состояния (Вера Наконечная) (гуглится как что-то связанное с UML) "--- }
    \item {Диаграмма ..... (Вера Наконечная) "--- }

    \textbf{И}
    \item {Интерн (Александр Кузнецов) "--- }

    \textbf{К}
    \item {Компьютерная Безопасность (Игорь Юрин) "--- }

    \textbf{Л}
    \item {Логирование (Барабанов Никита) "---}

    \textbf{М}
    \item  {Маркетинговый аналитик (Шлюпкин Павел) "--- }
    \item {Микросервис (Барабанов Никита)"---}
    \item {Микросервисная архитектура (Барабанов Никита) "---}
    \item {Монолитная архитетура (Барабанов Никита) "--- }
    \item {Мониторинг (Барабанов Никита) "--- }
    
    \textbf{О}
    \item {Оффер (Барабанов Никита) "---}
    
    \textbf{П}
    \item {Продуктовый аналитик (Шлюпкин Павел) "---}
    
    \textbf{Р}
    \item {Реляционная база данных (Вера Наконечная) "---}

    \textbf{С}
    \item {Среды разработки (Михаил Ильичев) "--- }
    \item {Спецификации (Вера Наконечна) "--- справочник методов}
    \item {Синхронная ..... (Вера Наконечная) "---}
    
    \textbf{Т}
    \item {Тестировщик (Александр Кузнецов) "--- }
    \item {Тестирование (Константин Ланских) "--- }
    \item {Троян (Игорь Юрин) "--- }
    \item {ТЗ (Барабанов Никита) "--- Техническое задание "--- (если надо более развернутое определение)}

    \textbf{Х}
    \item {Хакер (Игорь Юрин) "---}

    \textbf{Э}
    \item {Экстримальное программирование (Барабанов Никита) "---}

\end{enumerate}

\section{Лекции №1, 2. Александр Кузнецов.}
\begin{enumerate}
    \item outsource "--- условный найм людей из других регионов / стран для экономии над з.п.
    \item Lead "--- ведущий разработчик программного обеспечения
    \item Cheif "---
    \item Senior "---
    \item Junior "---
    \item Middle "---
    \item Интерн "---
    \item ИИ "--- Искуственный Интелект
    \item Тестировщик "---
    \item Back"=end "---
    \item Front"=End "---
    \item Архитектор "---
    \item Кеш "---
    \item База Данных "---
    \item .NET framework
    \item API "--- программный интерфейс для предложения
\end{enumerate}

\section{Лекция №3. Игорь Юрин.}
\begin{enumerate}
    \item Компьютерная Безопасность "---
    \item Вайпер "--- программа для очищения данных
    \item Wi"=Fi (Вай-фай) "---
    \item Троян "---
    \item Хакер "---
    \item wipe "--- средство невосстановимого удаленя информации на носителе
    \item proksi сервер "---
\end{enumerate}

\section{Лекция №4. Шлюпкин Павел}
\begin{enumerate}
    \item биг тех "--- 
    \item таски "--- 
    \item project manager "---
    \item product manager "--- 
    \item scrum master "---
    \item agile coach "---
    \item AI"=тренер "---
    \item xr "---
    \item vr "---
    \item DevRel "---
    \item дата аналитик "---
    \item бизнес и системный аналитик "---
    \item продуктовый аналитик "---
    \item маркетенговый аналитик "---
    \item ux/ui "---
\end{enumerate}

\section{Лекция №5. БараБанов Никита}
\begin{enumerate}
    \item soa "--- метод разработки ........
    \item оффер "--- 
    \item msa "--- 
    \item framework "--- 
    \item логирование "---
    \item микросервис "---
    \item микросервисная архитектура "---
    \item мнолитная архитектура "---
    \item ui "--- user interface "---
    \item business logic "---
    \item data interface "---
    \item database "---
    \item RESTful API (delete post put get) "---
    \item json "--- js option natation "---
    \item devops "---
    \item pipeline "---
    \item unit test "---
    \item auto test "---
    \item deploy "--- запуск проекта на платформу (на hosting, beta test)
    \item production (prod) "---
    \item predproduction (predprod) "---
    \item push "--- отправить
    \item eml(uml???????) "---
    \item мониториг "---
    \item тз (техническое задание) "--- 
    \item waterfall model "---
    \item agile model "---
    \item scrum "---
    \item demo "--- предпросмотр решения
    \item экстремальное программирование "--- каждый аспект возводится в  абсолют
    \item kaizen "--- регулярное совершенствование 
\end{enumerate}

\section{Лекция 6. Михаил Ильичев}
\begin{enumerate}
    \item среды разработки "--- 
    \item git "---  система контроля версий
    \item unit"=test "---  
    \item SRE "--- 
\end{enumerate}

\section{Лекция 7. Иван Жадаев}
\begin{enumerate}
    \item sql запросы
\end{enumerate}


\section{Лекция 8. Вера Наконечная}
\begin{enumerate}
    \item диограмма состояния "---
    \item диаграмма .... "--- 
    \item алгоритмическое мышление "---
    \item брокер сообщений "---
    \item swagger "--- один из часто используемых мини-справочников методов(спецификации)
    \item спецификации "--- справочник методов
    \item синхроная .... "---
    \item ассинхронная .. "---
    \item реалиционные бд "---
    \item nosql бд "---
    \item веб-API "---
\end{enumerate}

\section{Лекция 9. Александра Крючкова/Константин Ланских/Антона Пудикова}
\begin{enumerate}
    \item тестирование "---
    \item тестировщики "---
    \item sft инженеры (те кто тестят ручками) "---
\end{enumerate}
\end{document}
